\chapter{Introducere}
\section{Descriere}
    \LARGE{Scurtă descriere a algorimului implemantat de mine:}
    \begin{itemize}
        \item \LARGE{preia un vector dintr-un fisier}
        \item \LARGE{creeaza un număr de găleți egal cu câtul împărțiri valori maxime a vectorului la zece adunat cu 1 }
        \item \LARGE{elementele se inserează în galeata cu numărul egal cu câtul împărțiri valori absolute a elementului la zece}
        \item \LARGE{elementele din fiecare găleata sunt sortate cu algoritmul de sortare prin insertie}
        \item \LARGE{gălețile sunt concatenate}
    \end{itemize}


\subsection{Exemple}
\begin{enumerate}
    \item {
     \rect{10} \rect{1.1} \rect{1} \rect{5} \rect{0} \rect{3.2} \\
     \rect{1.1 , 1 , 5 , 0 , 3.2} \rect{10}\\
     \rect{0 , 1 , 1.1 , 3.2 , 5} \rect{10}\\
     \rect{0 , 1 , 1.1 , 3.2 , 5 , 10}
    }
    \vskip 0.5cm
    \item {
     \rect{14} \rect{11} \rect{1} \rect{5} \rect{0} \rect{3.2} \rect{21} \rect{20} \\
     \rect{1 , 5 , 0 , 3.2} \rect{14 , 11} \rect{21 , 20 }\\
     \rect{0 , 1 , 3.2 , 5} \rect{11 , 14} \rect{20 , 21}\\
     \rect{0 , 1 , 3.2 , 5 , 11 , 14 , 20 , 21}
    } 
\end{enumerate}

\vskip 0.5cm

\section{Pseudocod}
\nocite{Geeks}
\begin{lstlisting}
n:=lungime(x);
for i:=1 to n do
insereaza x[i] in lista 
GAL [parte intreaga(n*x[i])]
for i:=0 to n - 1 do
 sorteaza lista GAL[i] folosind
 sortarea prin insertie
concateneaza in ordine listele 
GAL [0] , GAL[1], ... ,GAL[n - 1]
\end{lstlisting}